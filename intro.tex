\section{Introduction}

In recent years, distant supervision (\DS) is emerging as a viable means for supporting various classification tasks -- from relation extraction~\cite{mintz2009distant} and sentiment classification~\cite{go2009twitter} to cross-lingual semantic
analysis~\cite{fang2016learning}. 
\orange{By distilling feasible and effective rules from prior knowledge, this technique applys these rules to automatically prepare large scalse training data for these tasks.} 
% \orange{In the context of relation extraction}, \DS works by aligning  (\texttt{subject}, \texttt{relation}, \texttt{object}) triples in knowledge base 
% automatically aligning facts from knowledge bases to text, this technique allows us to scale a small number of seeds to millions of instances.


%. In the
%context of relation extraction, one could start by feeding \DS with a set annotated texts where each
%sentence is mapped to a relation, i.e., a fact tuple like $<$PERSON, \texttt{born-in}, PLACE$>$; \DS can then use the knowledge
%extracted from the texts to
%generate relation contexts from unlabeled texts.

While promising, \DS does not guarantee perfect results and often introduces noise to the
generated data. In the context of relation extraction, \DS could match the \orange{knowledge base triple} $<$\emph{Donald Trump},
\texttt{born-in}, \emph{New York}$>$  in \emph{false positive} contexts like \emph{Donald Trump worked in New York City}.
\orange{Prior works~\cite{takamatsu2012reducing,ritter2013modeling}} show that \DS often mistakenly labels real positive instances as negative (\emph{false negative}) or
versa vice (\emph{false positive}), and there could also have confusions among positive labels too. These noises can
severely affect the training
procedure and lead to poorly-performing models.
%This drawback greatly hinders the wide adoption of this powerful technique.

However, it is not trivial to tackle the noisy data problem of \DS, since there is usually no explicit supervision to guide us for capturing the noise.
%There have been attempts to tackle the noisy data problem of \DS.
Previous works have tried to remove sentences containing unreliable syntactic patterns~\cite{takamatsu2012reducing}, design new models \orange{under the} 
% \textbf{\textit{at-least-one assumption}}  
\textit{at-least-one assumption}
that at least one of the aligned sentences supports the triple in knowledge base~\cite{riedel2010modeling}, where they either introduce new variables in probabilistic graphic models to capture certain types of noise, or aggregate predictions from multiple classifiers to reduce the influence of certain noise~\cite{hoffmann2011knowledge,surdeanu2012multi,ritter2013modeling,min2013distant}. These works represent a substantial leap forward towards making \DS more practical. However, these methods are either designed specifically to deal with certain types of noise, 
% e.g., false positive or false negative, 
or rely on manual rules to filter noise, thus are unable to scale.
On the other hand, recent breakthrough in neural networks~\cite{lin2016neural} show that neural network models can reduce the influence of incorrectly labeled data without explicitly characterizing the inherent noise by aggregating the multiple training instances attentively for relation classification.
% On the other hand, recent breakthrough in neural networks~\cite{lin2016neural} show that neural network models can make use of the noisy data without explicitly characterizing the inherent noise by aggregating the multiple training instances attentively for relation classification.
% On the other hand, recent breakthrough in neural networks show that neural network models can attentively aggregate multiple evidences to learn \orange{prominent representations (a little strange)} for relation extraction, and avoid characterizing the inherent noise explicitly~\cite{zeng2015distant,lin2016neural}. 
\orange{In fact, we find that the neural network architecture provides opportunities to explicitly characterize the noise underlying in the. This enables the model to infer the true label of incorrectly labeled data and thus can make use of rather than trying to ignore incorrectly labeled data.}


% The recent breakthrough in neural networks provides a new way to attentively aggregate multiple evidences to learn \orange{prominent representations (a little strange)} for relation extraction, and avoid characterizing the inherent noise explicitly. \cite{zeng2015distant} and \cite{lin2016neural} are among the first efforts in this direction, but their have \todo{a} significant shortcoming -- \todo{xx}.
% As shown in this paper, \todo{the assumption of xx is too strong in practice. As a result, they xxx.}

%In fact, those neural network architectures provide more opportunities to explicitly characterize the noise underlying in the data.
% \red{---F: This argument is weak! can we say they either rely on predefined rules or extra labeled dataset?}

In this paper, we show that while noise is inevitable in the \DS-style training data, it is still possible to characterize its pattern  in a unified framework along with its original classification objective. Our key insight is that the \DS-style training  data would typically contain useful clues about the noise pattern. For example, 
\orange{since some people work in their birth place, we can reasonably assume that a training sentence describing the work place of a person has some chances to be erroneously labeled by \DS as relation \texttt{born-in}}.
% if we see a training sentence describing the work place of a person, we can reasonably assume that it has some chances to be erroneously labeled by \DS as relation \red{$live\_place$} .
Accordingly, we propose to dynamically generate a transition matrix for each training instance to characterize the possibility that the \DS labeled relation is confused, to indicate its \orange{noise pattern}.  To tackle the challenge of no direct guidance over the noise pattern, we employ a curriculum learning based training method to gradually model the noise pattern over time \orange{and further use trace regularization to control the behavior of the transition matrix during training. Furthermore,} our novel approach also provides the flexibility to combine the prior knowledge of data quality. 
% \orange{by building a curriculum  utilizing trace regularization to control the behavior of the transition matrix during training.??}
We evaluate our method in the relation extraction task with various settings on two benchmark datasets. \orange{Experimental results show
that our transition matrix method consistently improves the model performance in all these settings, and the prior knowledge of data quality can further contribute to the training procedure and thus generate better results.}
% the proposed technique can better model the noise pattern over the state of the art \orange{(what is state-of-the-art? how about just say comsistently improves the model in various settings)}, leading to consistent improved performance.

%We evaluate our approach by applying it to \red{xxx??} and compare it against a
%\red{state-of-the-art xx}. Experimental results show that \red{our approach can create a
%higher-quality relation extraction classifier with xx\% better performance
%using the same set of training data. (----F: Maybe we should change another type of representation? or list our contributions here?)}

This paper makes the following specific contributions: \orange{(to be determined)}
\begin{itemize}
\item We propose to use dynamic transition matrix to characterize the noise introduced by DS, and extensively investigate the situation of relation extraction.
\item We propose a novel curriculum leanring based method for training, which also provides the flexibility to benefit from prior knowledge of data quality. 
\item Extensive experiments in two relation extraction datasets show that our method consistently improves the model performance in various settings.
\end{itemize}

