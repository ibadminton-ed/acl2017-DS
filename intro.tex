\section{Introduction}

In recent years, distant supervision (\DS) is emerging as a viable means for
supporting natural language processing (NLP) tasks such as
relation extraction~\cite{mintz2009distant}, xx~\cite{} and cross-lingual
semantic analysis~\cite{fang2016learning}.

\DS works by first extracting knowledge from a small set of labelled examples
and then use this knowledge to generate training examples from a large corpus
of data where manually labeling individual data items will be
expensive. In the context of relation extraction, one could start by
feeding \DS with a small number of annotated texts where each sentence is
mapped to a relation, i.e., a fact tuple like $<PERSON, born-in, PLACE>$;
\DS can then use the knowledge extracted from the texts to automatically
generate relation contexts from unlabeled texts.

One of the fundamental problems of \DS is that the algorithm does not
guarantee perfect results and often introduces noise to the generated data.
For instance, \DS could match the fact tuple $<``Donald Trump", born-in,
``New York">$  in \emph{false positive} contexts like ``\emph{Donald Trump
worked in New York City}". Prior work has shown that \DS often mistakenly
labels real positive instances as negative (\emph{false negative}) or versa
vice (\emph{false positive}), and there could have confusions between
positive labels too. These noisy data can significantly affect the training
procedure and lead to poorly-performing models.

There have been attempts to tackle the noisy data problem of \DS. These
include work on xx, xxx, and xx. These works represent a substantial leap
forward towards making \DS more practical. However, prior work is shitty
because …

In this paper, we show that while noise is inevitable, it is possible to
accurately model the noise pattern. Our key insight is that the noise is not
random? We propose a novel approach to xxx. We apply our method to relation
extraction. To tackle xxx, we employ curriculum learning to xxx…. As
knowledge is built up, our algorithm can xxx.

We evaluate our approach by applying it to xxx and compare it against a
state-of-the-art xx. Experimental results show that our approach can create a
higher-quality relation extraction classifier with xx% better performance
using the same set of training data.
