\section{Problem Definition}
The task of distantly supervised relation extraction is to extract knowledge triples, $<$\emph{subj}, \texttt{rel}, \emph{obj}$>$, from free text with the training data constructed by aligning existing \KB triples with a large corpus.
\orange{Specifically, given a triple in \KB, \DS works by first retrieving all the sentences containing both \emph{subj} and \emph{obj} of the triple, and then constructing the training data by considering these sentences as support to the existence of the triple.}
This task can be conducted in both the sentence and the bag levels.
The former  takes a sentence $s$ containing both $subj$ and $obj$ as input, and outputs the relation expressed
by the sentence between $subj$ and $obj$.
The latter setting
\orange{alleviates the noisy data problem by using the \textit{\textbf{at-least-one assumption}}}
% is based on the \textit{\textbf{at-least-one assumption}}
that at least one of the retrieved sentences containing both $subj$ and $obj$ supports the $<$\emph{subj}, \texttt{rel}, \emph{obj}$>$ triple.
It takes a bag of sentences $S$ as input where each sentence
$s\in S$ contains both $subj$ and
$obj$, and outputs  the relation between $subj$ and $obj$ expressed by this bag.


%\blue{DS is a powerful technique for performing relation extraction on a large text corpus where manually labeling every single sentence is impossible}
%
%Distantly supervised relation extraction aims at extracting $<$ $subj$,  $rel$,  $obj$ $>$ triples from free text using distantly obtained training data, which can be examined in two different settings, i.e., sentence level, and bag level. The former  takes a sentence $s$ containing both $subj$ and $obj$ as input, and output the relation expressed by the sentence between $subj$ and $obj$. The latter setting is based on the \textit{at-least-one} assumption (\todo{explain the assumption if not explained before}) and takes a bag of sentences $S$ as input where each sentence $s\in S$ contains both $subj$ and $obj$, and output  the relation expressed by the sentence bag between $subj$ and $obj$.
