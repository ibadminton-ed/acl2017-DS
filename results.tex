\section{Experimental Results \label{sec:evaluation}}

\subsection{Performances on \TimeRE}
\paragraph{Sentence Level Models}
The results of sentence level models on \TimeRE are shown in Figure \ref{fig:
sent_luo}. We can see that mixing all subsets together (\texttt{sent\_mix})
gives the worst result. The performance of this strategey is significantly
worse than using the reliable subset only (\texttt{sent\_reliable}). This
suggests the noisy nature of the training data obtained through \DS and
properly dealing with the noise is the key for \DS to be adopted at a wider
scale. When getting help from our transition matrix during training, the
model (\texttt{sent\_mix\_TM}) significantly improves (\texttt{sent\_mix}),
delivering the same level of performance as (\texttt{sent\_reliable}) in most
cases. This suggests that our transition matrix can help to mitigate the bad influence of noisy training instances.


Let us consider the \texttt{CL} scenario where one can first train on the
reliable subset and then gradually move to the space with both reliable and  less
reliable subsets. In this case, the curriculum learning based model
(\texttt{sent\_CL}) even  outperforms \emph{sent\_reliable} significantly \red{(F: does \texttt{sent\_CL} have help from other components? e.g., TM?) },
indicating that the curriculum learning framework not only reduces the effect
of noise, but also helps the model learns from noisy data. When applying the
transition matrix approach into the curriculum learning framework using one reliable
subset and one less reliable one,  \blue{ our model (\emph{sent\_CL\_seg2\_TM})
improves \emph{sent\_CL} by  %exploring the prior knowledge about data quality and
enabling the transition matrix approach to control the noise at different
levels during different curriculums. }  \todo{ZW: Have no idea of what this sentence is talking about. \textbf{I have rephrased it}.}
It is not surprising that when we use more less reliable data, i.e., using all
three subsets, our model (\emph{sent\_CL\_TM}) significantly outperforms all
other models by a large margin\footnote{We will use all three subsets for all
\emph{\_CL\_TM} settings in the rest of the experiments.}. \todo{ZW: I
rephrase some of the sentences, but I still think this paragraph needs to be
written.}


\paragraph{Bag Level Models}
In this experiment, we first look at the performance of the bag level models with attention aggregation. The results are shown in Figure \ref{fig: bag_att_luo}.
Consider a comparison between the  model trained on the reliable subset only (\texttt{bag\_att\_reliable}) and  the model trained on the mixed dataset (\texttt{bag\_att\_mix}).
In contrast to the sentence level cases, \texttt{bag\_att\_mix} outperforms \texttt{bag\_att\_reliable} by a large margin. This is due to the fact that  \texttt{bag\_att\_mix} has taken the noise within the bag into consideration through the attention aggregation mechanism, which can be seen as a denoising step within the bag.
This may also be the reason that when we introduce either our transition matrix approach \red{(\texttt{bag\_att\_TM})}  or curriculum learning framework (\texttt{bag\_att\_CL})   into the bad level model , the improvement compared to \texttt{bag\_att\_mix}  is not as significant as in the sentence level.
However, when we utilize our transition matrix approach to control the noise ate different levels within the curriculum learning paradigm (\texttt{bag\_att\_CL\_TM}), the performance gets further improved. This is especially in the high precision part compared to \texttt{bag\_att\_CL}.
We also note that the bag level's  \textit{at-least-one assumption} does not always hold, and there are still false negative and false positive problems. Therefore, we can see that using our transition matrix approach with  or without curriculum learning, i.e.,  \texttt{bag\_att\_TM}  and \texttt{bag\_att\_CL\_TM}), both improve the performance, and \texttt{bag\_att\_CL\_TM} performs slightly better.

%\paragraph{Bag Level Average Aggregation Models}
The results of the bag level models with average aggregation is shown in Figure \ref{fig: bag_avg_luo}. The relative ranking of various settings is similar to those with the attention aggregation. \red{why (\texttt{bag\_avg\_reliable}) performs so much better than all others in the very high precision stage ($recall<0.05$)?????????} One of the notable differences is that both \texttt{bag\_avg\_CL} and \texttt{bag\_avg\_TM} improve \texttt{bag\_avg\_mix} with larger margins compared to that in the attention aggregation setting. The reasons may be that the average aggregation mechanism is not as good as the attention aggregation one in terms of denoising ability, which leaves more space for our transition matrix approach or curriculum learning framework to improve.
\blue{Another prominent difference lies in the performance drop of \emph{bag\_avg\_CL\_TM} in the low-recall area ($recall<0.15$)., why???}
%However, since  denoising ability is not as good as attention aggregation, adding unreliable data gradually (\emph{bag\_avg\_currd}) improves the model performance here. We can also see that the transition matrix improves the average aggregation models more significantly than the attention aggregation models.
 %Note that due to the inferior denoising ability of average aggregation, the unhandled sentence level noise may further propagates to bag level, which gives the transition matrix more chance to help model the noise.

\begin{figure}[htbp]
\begin{center}
\includegraphics[width=0.9\linewidth]{figures/best_cmp_exp_overall.png}
\caption{Comparison on TimeRE}
\label{fig: cmp_luo}
\end{center}
\end{figure}

\paragraph{Compare to the-state-of-the-art} \red{should change to something else??? e.g., case study, or static TM }
The comparison of the best settings of each model family is shown in Figure \ref{fig: cmp_luo}. We can see that all of our transition matrix models outperform a state-the-art model presented in \cite{luo2016temporal}. With the help of transition matrix, although the basic version of average aggregation is not as good as attention aggregation, its transition matrix version is similar to the attention aggregation. Also note that although the sentence level models trained on mixed data do not perform very good, the sentence level model can use transition matrix to model the sentence level noise and thus performs best in all these models. Recall that the transition matrix can model the noise rather than just reduce the influence of noisy sentences as in bag level models, the sentence level model actually has the ability to make use of the noisy data. This shows that sentence level noise is more significant than the bag level noise in relation extraction, and modeling noise works better than just trying to reducing the influence of noise.


\begin{figure}[t!]
\includegraphics[width=0.9\linewidth]{figures/re_att_avg_cmp_exp.png}
\caption{Results on Dataset of Riedel et.al.}
\label{fig: Riedel_res}
\end{figure}

\subsection{Performance on \EntityRE}
We also conduct experiments on the \EntityRE dataset, where we can evaluate our bag level models only. \todo{ZW: Why bag-level models only? We need a an answer! } \blue{F: I put a sentence in data preparation.}
We  implemented the average aggregation method (\texttt{avg}), and the attention aggregation method (\texttt{att}) proposed by \cite{lin2016neural} as well as their corresponding transition matrix versions (\texttt{avg\_TM} and \texttt{att\_TM}). As we can see in Figure \ref{fig: Riedel_res},  due to the inferior denoising ability of the average aggregation component, \texttt{avg} performs worst among all those models. %Similar to the trends on the TimeRE data, when
\red{what should we say about this figure????}
When we inject our transition matrix approach into both \emph{avg} and \emph{att}, the resulting two  models, both of which can  clearly outperform their basic extraction models.
%
%since the unhandled sentence level noise propagates to the bag level, which makes the bag level noise become more severe, the transition matrix has more chance to model the noise. Therefore, the \emph{avg\_TM} model clearly outperforms the \emph{avg} model.
Again, because the attention aggregation model , this model already has good ability in reducing the impact of sentence level noise. Since the bag level noise is less significant than the sentence level noise, the improvement of our transition matrix model is limited, which only improves the model on the low recall part.
%Note that the low recall part corresponds to high precision, which is more useful than the rest of the extraction results in practice. Therefore, our transition matrix method is also useful in this situation.

\iffalse
\begin{figure*}[htbp]
\centering
\subfigure[Overall PR Curves]{
\includegraphics[width=0.475\linewidth]{reg_exp_overall.png}
\label{fig: reg_overall_pr_curve}
}
\subfigure[Small Relation PR Curves]{
\includegraphics[width=0.475\linewidth]{reg_exp_small.png}
\label{fig: reg_small_rel_pr_curve}
}
\caption{Imapct of Regularization Weights}
\label{fig: reg_PR_curve}
\end{figure*}
\fi

\subsection{Summary}
\todo{ZW: This section is incredibly complex. I suggest to have a summary section to highlight the take away messages.} 