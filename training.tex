\section{Training Procedure \label{sec:training}}
One of the key innovations of this work is on how to train and generate the
transition matrix. The challenge here lies in the fact that we often do not
have direct guidance over the noise pattern. This means that simply aligning
the noisy label with the observed distribution, $\mathbf{o}$, does not
guarantee the prediction distribution, $\mathbf{p}$, will match the true
relation distribution. To tackle this challenge, we employ curriculum
learning to guide the algorithm to gradually learn how to model the noise,
rather than modelling the noise at the very beginning of the training when
there is little knowledge of the noise pattern. To \todo{ZW: prevent the
transposition matrix becomes bias towards xx??}, we use  trace regularization
to control the behavior of transition matrix. Our approach also provides the
flexibility to combine any prior knowledge of noise to improve the
effectiveness of  the transition matrix. We show that if one could break the
data into reliable and unreliable parts, our approach can exploit the split
as indirect supervision over the noise pattern to build a more effective
transition matrix to better model the noise. The sum is greater than its
parts. As can be seen latter in Section~\ref{sec:evaluation}, putting
together these techniques
enables us to build an adaptive scheme to better model the noise pattern over the state-of-the-art. 


%To handle this situation, instead of modeling the noise at the very beginning of training, we use curriculum learning to guide the model to gradually learn to model the noise, and use trace regularization to control the behavior of transition matrix. Furthermore, if we have prior knowledge to roughly separate the data into reliable and unreliable parts, we show that we can use the split as indirect supervision over the noise pattern to help the transition matrix to better model the nosie.
%Apart from that, we also show how to constrain the \red{ability??} of the transition matrix to avoid \red{overfitting???}.

\subsection{Trace Regularization}
Intuitively, if the noise is small, the transition matrix, $\mathbf{T}$, will be likely become an identity matrix (and vice versa).  Since each row of $\mathbf{T}$ sums to 1, the similarity between the transition matrix and the identity matrix \todo{ZW: what's an identity matrix?} can be represented by the trace of the transition matrix, $\mathbf{T}$. The larger the $trace(\mathbf{T})$ is, the smaller the number of elements that do not lie in the diagonal will be, and the more similar the transition matrix $\mathbf{T}$ to identity matrix is. Therefore, we can realize our expectation over the noise level of the data by controlling the value of $trace(\mathbf{T})$. \todo{ZW: Have no idea of what this paragraph is talking about...But it might be just me.}

\subsection{Curriculum Learning}
The idea of curriculum learning is simple: we start with the easiest aspect of a task, and then level up the difficulty gradually.
This algorithm fits well with our problem. 
There are situations where we have the knowledge of the data quality and situations where we do not. Curriculum learning performs
well in both scenarios.    


\paragraph{Without Prior Knowledge of Data Quality}
To handle the situation that $\mathbf{p}$ does not guarantee to match the true relation distribution, a straightforward method is to first train the model without noise modeling and then emphasis noise modeling gradually. In this way, the prediction branch is roughly trained before the model tries to model the noise. We implement this idea in the curriculum learning framework. Specifically, we consider ignoring the noise is the easier part of training compared with modeling the noise, and our loss function is designed as:

%If no guidance over the noise pattern exists, using only $\mathbf{o}$ to match the noisy label does not guarantee $\mathbf{p}$ to match the true relation distribution. Therefore, we use the linear combination of the cross entropy of both $\mathbf{o}$ and $\mathbf{p}$ as our objective function. Furthermore, we build a curriculum by controlling the training objective to gradually emphasis on noise modeling. Specifically, we design a decreasing weighting scheme for both the cross entropy of output prediction and the trace regularization component:
%\red{ gradually controlling the impact of the transition matrix}. Specifically, \red{we design a decreasing weighting scheme for the trace regularization component}, defined as:

\begin{equation}
\begin{aligned}
Loss	&=\sum_{i=1}^N{-((1-\alpha) log(o_{iy_{i}}) + \alpha log(p_{iy_{i}}))} \\
&+ \beta trace(\mathbf{T}_{i})
\end{aligned}
\label{general_loss}
\end{equation}
where $0\le\alpha\le1$, $y_i$ is the relation assigned by distant supervision for datum $i$, $o_{iy_{i}}$ and $p_{iy_{i}}$ are the probabilities that the observed and predicted relation for datum $i$ is $y_i$ respectively. Instead of only using the observed relation distribution $\mathbf{o}$ to simulate the relation labeled by distant supervision, we use the linear combination of the cross entropy of both the observed relation distribution $\mathbf{o}$ and the predicted relation distribution $\mathbf{p}$.

At the start of training, we set $\alpha=1$ and $\beta<0$, which means we do not expect to model the noise. As the training proceeds, the prediction branch gradually learns the basic prediction ability, then we increase the difficulty level by decreasing $\alpha$ and the absolute value of $\beta$ by $\rho$ every $\tau$ epochs to gradually guide our model to learn to model the noise. Since we do not make any assumptions about the data set, this method actually applies to any situations, and therefore we use this as our default training method.

\paragraph{With Prior Knowledge of Data Quality}
If we have prior knowledge about which part of the training data is more reliable and which is unreliable, we can use this knowledge as indirect guidance over the noise pattern by helping the model to distinguish reliable data from unreliable ones. Specifically, we can build a curriculum by first training the prediction branch on the reliable data for some epochs and then add the unreliable data to train the full model. In this way, the prediction branch is roughly trained before exposed to more noisy data.

Furthermore, we can also take better control of the training procedure by trace regularization.
%\red{utilize our prior knowledge of the data quality in the form of trace regularization}.
Specifically, our loss function is:

\begin{equation}
\begin{aligned}
Loss=\sum_{i=1}^M{\sum_{j=1}^{N_i}{-log(o_{ijy_{ij}})}} + \beta_i trace(\mathbf{T}_{ij})
\end{aligned}
\end{equation}
where the left side is vanilla cross entropy and the right side is trace regularization, $i$ is the index of the data subsets, $j$ is the index of training data, $\beta_i$ is the trace regularization weight for subset $i$, $\mathbf{T}_{ij}$, $y_{ij}$ and $o_{ijy_{ij}}$ are the transition matrix, relation labeled by distant supervision, and the observed probability of that relation for datum $j$ in subset $i$ respectively.

For the reliable subset, we want $trace(\mathbf{T})$ to be large (negative $\beta$) so that the element values of $\mathbf{T}$ will be centralized to the diagonal and the transition matrix will be similar to identity matrix. As for the unreliable subsets, we want the $trace(\mathbf{T})$ to be small (positive $\beta$) so that the element values of their transition matrices will be diffusive and the transition matrix will be less similar to identity matrix. In other words, the transition matrix is encouraged to model the noise. Note that this loss function only works for sentence level models, since reliable sentences and unreliable ones are all aggregated into a sentence bag in the bag level models and therefore we can not determine which bag is reliable and which is not. However, bag level models can still use the curriculum by changing the content of the bag and use Equation \ref{general_loss} for training. In this way, it can also benefit from the prior knowledge of data quality.



\subsection{Constrained Transition Matrix}
\orange{Since the triples in knowledge base are reliable in most of the times, the positive label confusion noise is less likely than the false negative and false positive noise.} However our transition matrix has the ability to model all these three types of noise. To prevent overfitting and make the model \red{concentrate on the false negative and false positive noise??}\orange{(not sure about the problem)}, we restrict the transition matrix for bag level models so that only the diagonal, the first column and the first row of the transition matrix do not equal to zero (assume the index of \emph{no-relation} is 0).
